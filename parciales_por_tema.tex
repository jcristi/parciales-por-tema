\documentclass[12pt,a4paper]{article}
%\usepackage[latin1]{inputenc}
\usepackage[utf8]{inputenc}
\usepackage[spanish]{babel}
\usepackage{amsmath}
\usepackage{amsfonts}
\usepackage{amssymb}
\usepackage{graphicx}
\usepackage[dvipsnames]{xcolor}
\usepackage{enumitem}
\usepackage{hyperref}

\author{Juan Cristi \\ email \href{mailto:jcristi@fi.uba.ar}{jcristi@fi.uba.ar}} % Si quieren poner como autor al Dto. de Matemática - FIUBA, no hay problema.
\title{Ejercicios de parcial por tema}

\renewcommand{\b}[1]{\textbf{#1}}

\newcommand{\red}[1]{\textcolor{red}{#1}}
\newcommand{\blue}[1]{\textcolor{blue}{#1}}
\newcommand{\green}[1]{\textcolor{green}{#1}}

\newcommand{\R}{\mathbb{R}}
\newcommand{\y}{\text{ y }}

\begin{document}
\maketitle
Esta es una recopilación de ejercicios de parcial de la materia Análisis Matemático II de la FIUBA ordenados por tema. \\

\red{En rojo figuran comentarios a los ejercicios.}\\

\blue{Los ejercicios en azul son la versión en el tema 2 de ejercicios anteriores, por lo que pueden obviarse. Los dejo para el que quiera entender cuanto varían los ejercicios cuando cambia el tema.}\\

Al inicio de cada ejercicio figura la fecha del parcial, el tema y el número de tal ejercicio.\\

\b{Si encontrás errores en este documento no dudes en escribirme a \href{mailto:jcristi@fi.uba.ar}{jcristi@fi.uba.ar}.}

\b{Encontrá siempre la versión más actualizada y corregida de este documento en \url{https://github.com/jcristi/parciales-por-tema/blob/master/parciales_por_tema.pdf}}


\section{TPII}
\subsection{Dominio, conjuntos de nivel, y otros conjuntos definidos a partir de funciones}\label{dominio}
\begin{enumerate}
	\item (10/12/19-T1-2) Dada $f(x, y)=\frac{\sqrt{x^2-x}}{\sqrt{xy-x}}$, \textbf{determine} y \textbf{grafique} su dominio natural $D$ y el conjunto de nivel $1$ de $f$.
	
	\item \blue{(10/12/19-T2-2) Dada $f(x, y)=\frac{\sqrt{y^2-y}}{\sqrt{xy-y}}$, \textbf{determine} y \textbf{grafique} su dominio natural $D$ y el conjunto de nivel $1$ de $f$.}
	
	\item (21/11/18-T?-3) \red{Este ejercicio incluye temas del TPIII} Dada $f: \mathbb{R}^2\rightarrow\mathbb{R}/ f(x, y)=2x^2y-y^2$, \textbf{determine} y \textbf{grafique} el conjunto $H$ del plano $xy$ en cuyos puntos la función $h=f'_x f'_y$ resulta con valores positivos y \textbf{analice} si $H$ es conexo. \label{ej:parciales3}
	
	\item (19/11/19-T?-2) Dada $f(x, y)=\frac{\sqrt{(x-1)y}}{\sqrt{x-y}}$, \textbf{determine} y \textbf{grafique} su dominio natural $D$ y el conjunto de nivel $1$ de $f$.	

	\item (16/11/19-T1-2) Dada $f(x, y)=\frac{\sqrt{xy-x^3}}{\sqrt{x^2+y-2}}$, \textbf{determine} y \textbf{grafique} su dominio natural $D$ e \b{indique} un ejemplo de punto interior a $D$ y dos de punto frontera de $D$ (Uno perteneciente y otro que no pertenezca a $D$).

	\item \blue{(16/11/19-T2-2) Dada $f(x, y)=\frac{\sqrt{xy-y^3}}{\sqrt{x+y^2-2}}$, \textbf{determine} y \textbf{grafique} su dominio natural $D$ e \b{indique} un ejemplo de punto interior a $D$ y dos de punto frontera de $D$ (Uno perteneciente y otro que no pertenezca a $D$).}
	
	\item (21/10/19-T1-2) Sea $f(x, y)=\frac{\ln{(4-x^2-y^2)}}{y^2-x^2}$ definida en su dominio natural $D$. \textbf{Determine} y \textbf{grafique} $D$ y el conjunto de nivel $0$ de $f$. \label{ej7}
	
	\item[\ref{ej7}-bis.] (Este ejercicio surgió como un error al copiar el ejercicio \ref{ej7}. Lo dejo por compatibilidad con versiones anteriores) Sea $f(x, y)=\frac{\ln{(xy-x^3)}}{y^2-x^2}$ definida en su dominio natural $D$. \textbf{Determine} y \textbf{grafique} $D$ y el conjunto de nivel $0$ de $f$.
	
	\item (19/10/19-T1-2) \red{Este ejercicio incluye temas del TPIII} Sea $\bar{f}: D\subset \R^2\rightarrow\R^2 / \bar{f}(x, y) = (\ln{(y-x)}, \ln(x^2-xy))$ donde $D$ es su dominio natural. \textbf{Determine} y \textbf{grafique} $D$ y el conjunto de puntos $H$ para los cuales $\bar{f}'_x(x,y)+\bar{f}'_y(x,y)=(0,-1)$. \label{ej:parciales8}
	
	\item \blue{(19/10/19-T1-2) \red{Este ejercicio incluye temas del TPIII} Sea $\bar{f}: D\subset \R^2\rightarrow\R^2 / \bar{f}(x, y) = (\ln{(y-x)}, \ln(y^2-xy))$ donde $D$ es su dominio natural. \textbf{Determine} y \textbf{grafique} $D$ y el conjunto de puntos $H$ para los cuales $\bar{f}'_x(x,y)+\bar{f}'_y(x,y)=(0,1)$.} \label{ej:parciales9}
	
	\item (02/07/19-T1-4) Sea $f(x, y)=\sqrt{\ln{(2x+y-1)}}$ definida en su dominio natural $D$. \textbf{Determine} y \textbf{grafique} $D$ y el conjunto de puntos $S$ donde resulta $f^2(x,y)<\ln(3)$.
	
	\item \blue{(02/07/19-T2-4) Sea $f(x, y)=\sqrt{\ln{(x+2y-1)}}$ definida en su dominio natural $D$. \textbf{Determine} y \textbf{grafique} $D$ y el conjunto de puntos $S$ donde resulta $f^2(x,y)<\ln(3)$.}
	
	\item (10/06/19-T?-4) Dada $f:D\subset\R^2\rightarrow\R / f(x, y)=\sqrt{xy-y}$ definida en su dominio natural $D$, \textbf{determine} y \textbf{grafique} $D$ y el conjunto de puntos $M$ de la curva de ecuación $y=x^2+1$ donde $f$ queda definida.
	
	\item (08/06/19-T1-4) Dada $f:D\subset\R^2\rightarrow\R / f(x, y)=\ln{(x+y^2)}$ definida en su dominio natural $D$, \textbf{determine} y \textbf{grafique} $D$ y el conjunto de puntos $M$ de la curva de ecuación $x=y^2-21$ donde $f$ queda definida.
	
	\item \blue{(08/06/19-T2-4) Dada $f:D\subset\R^2\rightarrow\R / f(x, y)=\ln{(y+x^2)}$ definida en su dominio natural $D$, \textbf{determine} y \textbf{grafique} $D$ y el conjunto de puntos $M$ de la curva de ecuación $y=x^2-21$ donde $f$ queda definida.}
	
	\item (13/05/19-T?-1) \red{Este ejercicio incluye temas del TPIII} Dada $ f(x, y)=\sqrt{4-x^2-y^2}$ definida en su dominio natural $D$, \textbf{determine} y \textbf{grafique} $D$ y el conjunto de puntos $H$ para los cuales $f'_y(x,y)=1$. \label{ej:parciales15}
	
	\item (11/05/19-T1-1) \red{Este ejercicio incluye temas del TPIII} Dada $ f(x, y)=\ln{(x^2y)}$ definida en su dominio natural $D$, \textbf{determine} y \textbf{grafique} $D$ y el conjunto de puntos $H$ para los cuales $f''_{xx}(x,y)-f''_{yy}(x,y)=0$. \label{ej:parciales16}
	
	\item \blue{(11/05/19-T2-1) \red{Este ejercicio incluye temas del TPIII} Dada $ f(x, y)=\ln{(xy^2)}$ definida en su dominio natural $D$, \textbf{determine} y \textbf{grafique} $D$ y el conjunto de puntos $H$ para los cuales $f''_{xx}(x,y)-f''_{yy}(x,y)=0$.} \label{ej:parciales17}
	
	\item (11/12/18-T1-4) Dada $f:D\subset\R^2\rightarrow\R / \vec{f}(x, y)=(\ln{(y-x^2)}, \sqrt{9-y-x^2})$ donde $D$ es el dominio natural de $\vec{f}$, \textbf{determine} y \textbf{grafique} el mencionado dominio $D$ y el conjunto  $H$ en cuyos puntos alguna de las componentes del campo resulte nula.
	
	\item \blue{(11/12/18-T2-4) Dada $f:D\subset\R^2\rightarrow\R / \vec{f}(x, y)=(\ln{(x-y^2)}, \sqrt{9-x-y^2})$ donde $D$ es el dominio natural de $\vec{f}$, \textbf{determine} y \textbf{grafique} el mencionado dominio $D$ y el conjunto  $H$ en cuyos puntos alguna de las componentes del campo resulte nula.}
	
	\item (17/11/18-T1-1) \red{Este ejercicio incluye temas del TPIII y del TPV} Siendo $f(x,y)=x^3-6xy+3xy^2$ con $(x,y)) \in \R^2$, \b{grafique} el conjunto de puntos $H^+$ del plano $xy$ para el cual el determinsnte hessiano $H(x,y)$ de $f$ resulta positivo \green{y \b{analice} en cuáles de dichos puntos la función $f$ produce extremo local, \b{clasificándolo} y \b{calculando} su valor.}  \red{No resolver la parte en verde hasta llegara a extremos en el TP V.}
	
	\item \blue{(17/11/18-T2-1) \red{Este ejercicio incluye temas del TPIII y del TPV} Siendo $f(x,y)=y^3-6xy+3yx^2$ con $(x,y)) \in \R^2$, \b{grafique} el conjunto de puntos $H^+$ del plano $xy$ para el cual el determinsnte hessiano $H(x,y)$ de $f$ resulta positivo \green{y \b{analice} en cuáles de dichos puntos la función $f$ produce extremo local, \b{clasificándolo} y \b{calculando} su valor.}  \red{No resolver la parte en verde hasta llegara a extremos en el TP V.}}
	
	\item (22/10/18-T?-5) Dada $f:D\subset\R^2\rightarrow\R / f(x, y)=\frac{\sqrt{x^2+y^2-1}}{\ln{(y-x)}}$, donde $D$ es el dominio natural de la función. \textbf{Determine} y \textbf{grafique} el dominio $D$ y el conjunto   de nivel $0$ de $f$. \b{Indique} un ejemplo de punto exterior y otro de punto frontera de $D$.
	
	\item (20/10/18-T1-1) Sea $f:D\subset\R^2\rightarrow\R / f(x, y)=\frac{\ln{(x-y)}}{\sqrt{x-1}}$, donde $D$ es el dominio natural de la función. \textbf{Determine} y \textbf{grafique} el dominio $D$ y el conjunto   de nivel $0$ de $f$. \b{Indique} un ejemplo de punto interior y otro de punto frontera de $D$.
	
	\item \blue{(20/10/18-T2-1) Sea $f:D\subset\R^2\rightarrow\R / f(x, y)=\frac{\ln{(y-x)}}{\sqrt{y-1}}$, donde $D$ es el dominio natural de la función. \textbf{Determine} y \textbf{grafique} el dominio $D$ y el conjunto   de nivel $0$ de $f$. \b{Indique} un ejemplo de punto interior y otro de punto frontera de $D$.}
\end{enumerate}

\subsection{Límites y continuidad} \label{cont}
\begin{enumerate}
	\item \red{Este ejercicio incluye temas del TPIII} Sea $f:\R^2 \rightarrow \R$ definida por:
	\[f(x,y)= \begin{cases}
	\frac{xy}{x^2+2y^2} & \text{ si } (x, y)\neq(0,0)\\
	0 & \text{ si } (x, y)=(0,0)
	\end{cases}\]
	\b{Analice} si $f$ es continua en $(0,0)$, en base a su conclusión \b{opine} con fundamento si la función puede tener derivadas parciales continuas en un entorno de dicho punto. \label{ej:cont1}
	
	\item (02/07/19-T1-3) \red{Este ejercicio incluye temas del TPIII} Siendo $ f ( x, y ) = ( e^{x^2 + y^2} - 1 ) /( x^2 + y^2 ) $ para $ ( x, y ) \neq ( 0, 0 ) $, \b{determine} $ f ( 0, 0 ) $ de manera que $ f $ resulte continua en el origen y, en ese caso, \b{analice} la derivabilidad de $ f $ en $ ( 0, 0 ) $ según distintas direcciones. \label{ej:cont2}
\end{enumerate}
	
\section{TPIII}
\subsection{Curvas}
\begin{enumerate}
	\item (10/12/19-T1-3) \red{Este ejercicio se puede resolver más fácilmente con temas más avanzados} Sea $C$ la curva definida por la intersección de las superficies $\Sigma_1$ y $\Sigma_2$ cuyas ecuaciones son:
	\[\Sigma_1 : x^2 y + x z = 3 \text{  y  } \Sigma_2 : x y - z^2 = 1.\]
	Si \(r\) o es la recta tangente a \(C\) en \(P = ( 1, 2, 1 )\), \b{calcule}  la distancia entre los dos puntos \(A\) y \(B\) donde	r o interseca a la superficie de ecuación \(z = x^2 \).
	
	\item (19/11/19-T?-4) Sea $ C $ la curva de ecuación $ \vec{X} = ( t^2 + t, t - t^3, 2 t^4 ) $ con $ t\in \R $. Si $ r_A $ y $ r_B $ son las rectas tangentes a $ C $ en los puntos $ A = ( 2, 0, 2 ) $ y $ B = ( 0, 0, 2 ) $, \b{analice} si existe un plano que contenga a ambas rectas.
	
	\item (02/07/19-T1-1) Considere la curva $ C $ de ecuación $ \vec{X} = ( t^2 - t, 2 t + 2, t^2)$ con $ t\in \R$ y sea $ \Pi_o $ el plano normal a $ C $ en el punto $ ( 0, 4, 1 )  $. \b{Determine} cuál es el punto de $ \Pi_o $ más cercano al origen de coordenadas y \b{calcule}  la distancia desde dicho punto al origen.
	
	\item (08/06/19-T1-2) Sea la curva $ C $ incluida en la superficie $ \Sigma $ de ecuación $\vec{X} = ( u + v, u - v, u + v + 2 ( u - v ) 2 )$ con $ H ( -2, -1 ) =	( u, v ) \in\R^2  $. Sabiendo que la proyección de $ C $ sobre el plano $ xy $ tiene ecuación $ x + y = 1 $, \b{analice} si la recta tangente a $ C $ en $ ( 2, - 1, 4 ) $ tiene algún punto en común con el plano $ xz  $.
	
	\item (11/05/19-T1-2) Siendo $ C $ la curva de ecuación $ \vec{X} = ( t^4 - 2 t^2, t^3 - 2 t, t^3 - 3 t ) $ con $ t \in\R$, \b{halle} las ecuaciones de las	rectas tangentes a $ C $ en aquellos puntos donde dichas rectas son paralelas al “ eje $ y $” y \b{analice} si existe un plano que contenga a dichas rectas; en caso afirmativo \b{halle} una ecuación para ese plano.
	
	\item (20/10/18-T1-2) \red{Este ejercicio se termina de resolver con temas más avanzados.} Sabiendo que los puntos de la curva de ecuación $ X = ( t^2, 2 t + 1, 4 t ) $ con $ t\in\R $ pertenecen a la superficie de ecuación $ z = f ( x, y ) $ con $ f \in C^1 ( \R^2 )$, \b{halle} una aproximación lineal para $ f(0.98 ; 3.01)$ teniendo en cuenta que $ f'_x( 1, 3 ) = 5  $.
	
	\item (De 2018) Sea $ C $ una curva cuyos puntos pertenecen a la superficie de ecuación $ x^2 z -y^2 +z = 4  $. Sabiendo que la proyección ortogonal de $ C $ sobre el plano $ xy $ tiene ecuación $ y = x^2 $, \b{analice} si la recta tangente a $ C $ en $ (2, 4, z_0 ) $ interseca en algún punto al eje $ z $.
	
	\item (De 2018) Dada la curva $ C $ definida como intersección de las superficies $ \sigma_1 $ y $ \Sigma_2 $ cuyas ecuaciones son:
	\begin{align*}
		\Sigma_1 &: z = x^2 + y + 1 &\text{ con } (x, y) \in\R^2\\
		\Sigma_2 &: \vec{X} = (u, u^2, v) &\text{ con } (u, v) \in \R^2
	\end{align*}
	\b{analice} si la recta tangente a $ C $ en $ (2, y_0, z_0 ) $ tiene algún punto en común con el plano $ xz $.
	
	\item (De 2018) \b{Halle} el punto medio del segmento $\bar{AB}$, sabiendo que sus puntos extremos ($ A $ y $ B $) son aquellos donde la curva de ecuación $ \vec{X} = (t^2, 2 t, t^4 +4 t) $ con $t \in \R$ interseca a la superficie de ecuación $ \vec{X} = (u + v, u - v, 4 u v) $ con $(u, v) \in \R^2$.
	
	\item (De 2018) Sea $ C \subset\R^3 $ la curva incluida en la superficie $\Sigma$ de ecuación $ z = x y + 4  $ con $ (x, y)\in\R^2$. Sabiendo que una ecuación vectorial para $ C $ es $ X = (t - 1, t + 1, g(t)) $ con $ t\in\R$, \b{halle} una ecuación para el plano normal a $ C $ en $ (2, y_0, z_0 ) $.
	
\end{enumerate}

\subsection{Derivadas parciales}
De la sección \ref{dominio}, los ejercicios: \ref{ej:parciales3}, \ref{ej:parciales8}, \ref{ej:parciales9}, \ref{ej:parciales15}, 
\ref{ej:parciales16}, \ref{ej:parciales17}.\\

De la sección \ref{cont}, los ejercicios: \ref{ej:cont1}, \ref{ej:cont2}.\\

\begin{enumerate}
	\item (17/11/18-T1-2) Dada 
	\[f ( x, y ) = \begin{cases}
		\frac{\sin(2x^3+xy)}{x^2+y^2} & \text{ si } (x,y)\neq(0,0)\\
		0 & \text{ si } (x,y)=(0,0)
	\end{cases}\]
	Analice si $ f $ admite derivada parcial de 1º orden respecto de la variable $ x $ para todo $ ( x, y ) \in \R^2$.
	
	\item (De 2018) Sea $ f : \R^2 \rightarrow \R $ tal que $ f(x, y) = \sqrt{x^4 + y^4}  $. Analice si $ f $ admite derivada parcial de 1º orden respecto de la variable $ x $ en todo punto de su dominio.
\end{enumerate}

\subsection{Derivadas direccionales}
\begin{enumerate}
	\item (21/10/19-T?-5) Sabiendo que $ f ( x, y ) = \frac{x^2y}{x^2+2y^2} $ si $ ( x, y )\neq ( 0, 0 ) $ y que $ f ( 0, 0 ) = 0 $, \b{verifique} que $ f $ admite derivada direccional en toda dirección en $ ( 0, 0 ) $ y \b{determine} los versores para los que dicha derivada resulta nula.
	
	\item (13/05/19-T?-3) Siendo $ f ( x, y ) =	\frac{x \sin( x y )}{x^2 + y^2} $ si $ ( x, y ) \neq ( 0, 0 ) $ y $ f ( 0, 0 ) = 0 $, verifique que $ f $ admite derivada direccional en $ ( 0, 0 ) $ en toda dirección e indique si en base a lo obtenido se puede opinar con fundamento	acerca de la diferenciabilidad de $ f $ en el origen de coordenadas.
	
	\item (De 2018) Sea $ f : \R^2\rightarrow\R $ definida por:
	\[f(x, y) = \begin{cases}
		\frac{x^2\sin(y)}{x^2 + y^2} & \text{ si }(x, y)\neq(0, 0)\\
		0 & \text{ si }(x, y)=(0, 0)
	\end{cases}
	\]
	Analice la derivabilidad de $ f $ según distintas direcciones en el punto $(0, 0)$.
	
	\item (De 2018) Sea $ f : \R^2\rightarrow\R $ definida por:
	\[f(x, y) = \begin{cases}
	\frac{x\sin(xy)}{x^2 + y^2} & \text{ si }(x, y)\neq(0, 0)\\
	0 & \text{ si }(x, y)=(0, 0)
	\end{cases}
	\]
	Analice la derivabilidad de $ f $ según distintas direcciones en el punto $(0, 0)$.
\end{enumerate}

\subsection{Aproximación lineal}
\begin{enumerate}
	\item (16/11/19-T1-3) La curva de ecuación $ v = f (u ) $, en el plano $ uv $, tiene recta tangente de ecuación $ 2 u + v = 8 $ en el punto $ ( 2, v_0 ) $. Siendo $ h ( x, y ) = x^2 y + f (x, y) $, \b{calcule}  aproximadamente $ h ( 0.98; 2.03 ) $ mediante una aproximación lineal.
\end{enumerate}

\subsection{Plano tangente y recta normal}
\begin{enumerate}
	\item (10/12/19-T1-1) Sea $\Pi_0$ el plano tangente en $( 10, 2, 5 )$ a la superficie $\Sigma$ de ecuación $\vec{X}= ( u^2 + v, u - v, u + 2 v )$ con $( u, v ) \in \R^2$, \b{halle} una ecuación para la recta que contiene a los puntos donde $\Pi_0$ interseca a los ejes $x$ e $y$.
	
	\item (21/11/18-T?-4) La superficie $ \Sigma $ de ecuación $  z = x  $ y admite la recta normal $ r_0  $ en el punto $ A = ( 1, 2, z 0 ) $, dicha recta interseca a $ \Sigma $ en otro punto $ B $. Calcule la longitud del segmento cuyos puntos extremos son $ A $ y $ B $.
	
	\item (21/11/18-T?-5) Dada la superficie $ \Sigma $ de ecuación $ \vec{X} = ( u v, u + v, 2 u - v ) $ con $ ( u, v )\in\R^2$, \b{analice} si el plano tangente a $ \Sigma $ en el punto $ A = ( 6, 5, 1 ) $ tiene puntos en común con la curva $ C $ de ecuación $\vec{X}=\left(t^2, \frac{3 - t}{2}, \frac{21 + 7 t}{2} \right)$ con $ t \in\R $.
	
	\item (16/11/19-T1-1) Dada la superficie $ \Sigma $ de ecuación $ x^2 y + x z = 0 $, siendo $ r_A $ y $ r_B $ sus rectas normales en los puntos $ A = ( 1, - 1, 1 ) $ y $ B = (- 1, 1, 1 ) $ respectivamente, verifique que dichas rectas se intersecan en un punto y \b{halle} una ecuación para el plano que las contiene.
	
	\item (21/10/19-T1-1) Sea $ \Sigma $ la superficie de ecuación $ \vec{X} = ( u v, 2 u 2 + v, u - v ) $ con $( u, v )\in\R^2 $. \b{Halle} ecuaciones para la recta normal ($n_o$) y el plano tangente ($\Pi_o$) a $ \Sigma $ en el punto $ A = ( 2, 4, - 1 ) $ y \b{determine} los puntos de	$ n_o $ cuya distancia a $ \Pi_o $ resulte igual a $ 2 \sqrt{38}  $.
	
	\item (19/10/19-T1-3) Sea $ r_o $ la recta normal a la superficie de ecuación $ x = y z^2 + z $ en el punto $ ( 2, 1, 1 )  $. \b{Determine} los puntos en los que $ r_o $ interseca a los planos coordenados y \b{halle} la ecuación de un plano que contenga a dichos puntos y al origen de coordenadas.
	
	\item (08/06/19-T1-1) \red{Este ejercicio necesita temas de la guia 5 para resolverse completamente.} Dada la superficie $ \Sigma $ de ecuación $ z = f ( x, y ) $ con $ ( x, y ) \in\R^2$, donde $ f ( x, y ) = \frac{1}{2} x^2 y + \frac{1}{2} x^2 + y^2 + 8 $, \b{halle} los puntos de $ \Sigma $ en los que el plano tangente es horizontal (paralelo al plano $ xy $) y \b{analice} en	cuáles de dicho puntos el valor de $ f $ es un extremo local indicando, en ese caso, si es máximo o mínimo local.
	
	\item (11/12/18-T1-1) Sean $ C_1 \y C_2 $ dos curvas incluidas en la superficie regular $ \Sigma $. Sabiendo que ambas curvas contienen al punto $ A = ( 4, 4, 3 ) $ y que sus correspondientes ecuaciones vectoriales son:
	\[C 1 : \vec{X} = ( 2 t, t^2, 2 t - 1 )\text{ con } t \in\R \y C_2 : \vec{X} = ( 2 u - 2, u + 1, u ) \text{ con } u \in \R,\]
	\b{analice} si la recta normal a $ \Sigma $ en $ A $ tiene algún punto en común con el plano $ x + y + z = 27  $.
	
	\item (22/10/18-T?-2) Dada la superficie $ \Sigma $ de ecuación $ z = x^3y - 3 x^2y + y^2 + 4 $ con $ ( x, y )\in\R^2$, \b{halle} los puntos de $ \Sigma $ donde el plano tangente es horizontal (paralelo al plano $ xy  $) y \b{analice} si dichos puntos son colineales (pertenecen a la misma recta).
	
	\item (De 2018) Sabiendo que la recta normal a la superficie de ecuación $ z = f(x, y) $ en el punto $ (5, y_0, z_0 ) $ admite la ecuación $ \vec{X} = (1 + 2 u, 3 u, u + 4) $ con $ u \in\R $, \b{calcule}  una aproximación lineal para $ f(4.98; 6.01) $.
\end{enumerate}
	
\subsection{Gradiente y sus propiedades}
\begin{enumerate}
	\item (19/10/19-T1-4) Siendo $ f ( x, y ) = x + sen ( x y )  $, \b{halle} los versores $ \hat{r} $ tales que la derivada direccional de $ f $ en $ A = ( \sqrt{3}, 0 ) $ según $\hat{r}$ resulte igual al 50\% de la máxima derivada direccional en dicho punto.
	
	\item (10/06/19-T?-2) Sea $ r_o $ la recta tangente en $ ( 1, 2, 1 ) $ a la curva definida por la intersección de las superficies $ \Sigma_1 $ y $ \Sigma_2 $ de ecuaciones $ z + 4 = x^2 z + y^2 $ y $ x z^2 + y^2 = 5 $ respectivamente. \b{Determine} los puntos donde $ r_o $ interseca a la superficie de ecuación $ z^2 = x + y  $.
	
	\item (13/05/19-T?-2) Siendo $ C $ la curva definida por la intersección de las superficies  $ \Sigma_1 $ y $ \Sigma_2 $ cuyas ecuaciones son
	\[\Sigma_1 : 3 x y + \ln( x + y z - 6 ) - 2 z = 0 \text{ y } \Sigma_2 : z = x^2 y + e^{	x y - 2},\]
	\b{analice} si el plano normal a $ C $ en $ A = ( 1, 2, z_o ) $ tiene algún punto en común con el eje $ x  $.
	
	\item (17/11/18-T1-5) \b{Halle} el punto $ ( x 0, y 0, z 0 ) $ donde la recta definida por la intersección de los planos de ecuaciones $ x - y + z = 7 $ y $ 2 y - x + 3 z = 18 $ es normal a la superficie de ecuación $ z = x + y x^2  $.
	
	\item (22/10/18-T?-1) Siendo $ f : \R^3\rightarrow\R / f ( x, y, z ) = e^{(2x+y+z-7)}$, \b{analice} si la recta normal al conjunto de nivel 1 de $ f $ en $ ( 1, 2, 1 ) $ interseca en algún punto al plano de ecuación $ x = 3 z  $.
	
	\item (22/10/18-T?-4) Analice si la intersección de las superficies $\Sigma_1$ y $ \Sigma_2 $ de ecuaciones:
	\begin{align}
		\Sigma_1 :& z = 2 y 2 - x 2\\
		\Sigma_2 :& x \ln( y z ) + y \ln( x z ) = z - 1
	\end{align}
	define una curva $ C $ en un entorno del punto $ A = ( 1, 1, 1 )  $. En caso afirmativo, \b{determine} si existe algún punto en el que la recta tangente a $ C $ en $ A $ interseca al eje $ x  $.
	
	\item (De 2018) Sea $ f : \R^3\rightarrow \R $ con $ f \in C^1 (R^3)$. Sabiendo que $ X = (u, v^2, v) $ con $ (u, v) \in \R^2 $ es una ecuación de la superficie de nivel 1 de $ f $, que para $ A = (2, 1, 1) $ resulta $ f(A) = 1 $ y que la derivada direccional $ f'(A, \hat{r}) = 3 $ para $ \hat{r}=(1/\sqrt{ 3}, 1/\sqrt{ 3}, 1/\sqrt{ 3}) $, \b{calcule}  el valor de la mínima derivada direccional de $ f $ en el punto $ A $.
	
\end{enumerate}
\end{document}