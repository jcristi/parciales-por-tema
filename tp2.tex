\documentclass[12pt,a4paper]{article}
\usepackage[latin1]{inputenc}
\usepackage[spanish]{babel}
\usepackage{amsmath}
\usepackage{amsfonts}
\usepackage{amssymb}
\usepackage{graphicx}
\usepackage[dvipsnames]{xcolor}

\author{Juan Cristi}
\title{Ejercicios de parcial por tema}

\renewcommand{\b}[1]{\textbf{#1}}

\newcommand{\red}[1]{\textcolor{red}{#1}}
\newcommand{\blue}[1]{\textcolor{blue}{#1}}
\newcommand{\green}[1]{\textcolor{green}{#1}}

\newcommand{\R}{\mathbb{R}}

\begin{document}
\maketitle
Esta es una recopilaci�n de ejercicios de parcial de la materia An�lisis Matem�tico II de la FIUBA ordenados por tema. \\

\red{En rojo figuran comentarios a los ejercicios.}\\

\blue{Los ejercicios en azul son la versi�n en el tema 2 de ejercicios anteriores, por lo que pueden obviarse. Los dejo para el que quiera entender cuanto var�an los ejercicios cuando cambia el tema.}\\

Al inicio de cada ejercicio figura la fecha del parcial, el tema y el n�mero de tal ejercicio.

\section{TPII}
\subsection{Dominio, conjuntos de nivel, y otros conjuntos definidos a partir de funciones}
\begin{enumerate}
	\item (10/12/19-T1-2) Dada $f(x, y)=\frac{\sqrt{x^2-x}}{\sqrt{xy-x}}$, \textbf{determine} y \textbf{grafique} su dominio natural $D$ y el conjunto de nivel $1$ de $f$.
	
	\item \blue{(10/12/19-T2-2) Dada $f(x, y)=\frac{\sqrt{y^2-y}}{\sqrt{xy-y}}$, \textbf{determine} y \textbf{grafique} su dominio natural $D$ y el conjunto de nivel $1$ de $f$.}
	
	\item (21/11/18-T?-3) \red{Este ejercicio incluye temas del TPIII} Dada $f: \mathbb{R}^2\rightarrow\mathbb{R}/ f(x, y)=2x^2y-y^2$, \textbf{determine} y \textbf{grafique} el conjunto $H$ del plano $xy$ en cuyos puntos la funci�n $h=f'_x f'_y$ resulta con valores positivos y \textbf{analice} si $H$ es conexo. 
	
	\item (19/11/19-T?-2) Dada $f(x, y)=\frac{\sqrt{(x-1)y}}{\sqrt{x-y}}$, \textbf{determine} y \textbf{grafique} su dominio natural $D$ y el conjunto de nivel $1$ de $f$.	

	\item (16/11/19-T1-2) Dada $f(x, y)=\frac{\sqrt{xy-x^3}}{\sqrt{x^2+y-2}}$, \textbf{determine} y \textbf{grafique} su dominio natural $D$ e \b{indique} un ejemplo de punto interior a $D$ y dos de punto frontera de $D$ (Uno perteneciente y otro que no pertenezca a $D$).

	\item \blue{(16/11/19-T2-2) Dada $f(x, y)=\frac{\sqrt{xy-y^3}}{\sqrt{x+y^2-2}}$, \textbf{determine} y \textbf{grafique} su dominio natural $D$ e \b{indique} un ejemplo de punto interior a $D$ y dos de punto frontera de $D$ (Uno perteneciente y otro que no pertenezca a $D$).}
	
	\item (21/10/19-T1-2) Sea $f(x, y)=\frac{\ln{(xy-x^3)}}{y^2-x^2}$ definida en su dominio natural $D$. \textbf{Determine} y \textbf{grafique} $D$ y el conjunto de nivel $0$ de $f$.
	
	\item (19/10/19-T1-2) \red{Este ejercicio incluye temas del TPIII} Sea $\bar{f}: D\subset \R^2\rightarrow\R^2 / \bar{f}(x, y) = (\ln{(y-x)}, \ln(x^2-xy))$ donde $D$ es su dominio natural. \textbf{Determine} y \textbf{grafique} $D$ y el conjunto de puntos $H$ para los cuales $\bar{f}'_x(x,y)+\bar{f}'_y(x,y)=(0,-1)$.
	
	\item \blue{(19/10/19-T1-2) \red{Este ejercicio incluye temas del TPIII} Sea $\bar{f}: D\subset \R^2\rightarrow\R^2 / \bar{f}(x, y) = (\ln{(y-x)}, \ln(y^2-xy))$ donde $D$ es su dominio natural. \textbf{Determine} y \textbf{grafique} $D$ y el conjunto de puntos $H$ para los cuales $\bar{f}'_x(x,y)+\bar{f}'_y(x,y)=(0,1)$.}
	
	\item (02/07/19-T1-4) Sea $f(x, y)=\sqrt{\ln{(2x+y-1)}}$ definida en su dominio natural $D$. \textbf{Determine} y \textbf{grafique} $D$ y el conjunto de puntos $S$ donde resulta $f^2(x,y)<\ln(3)$.
	
	\item \blue{(02/07/19-T2-4) Sea $f(x, y)=\sqrt{\ln{(x+2y-1)}}$ definida en su dominio natural $D$. \textbf{Determine} y \textbf{grafique} $D$ y el conjunto de puntos $S$ donde resulta $f^2(x,y)<\ln(3)$.}
	
	\item (10/06/19-T?-4) Dada $f:D\subset\R^2\rightarrow\R / f(x, y)=\sqrt{xy-y}$ definida en su dominio natural $D$, \textbf{determine} y \textbf{grafique} $D$ y el conjunto de puntos $M$ de la curva de ecuaci�n $y=x^2+1$ donde $f$ queda definida.
	
	\item (08/06/19-T1-4) Dada $f:D\subset\R^2\rightarrow\R / f(x, y)=\ln{(x+y^2)}$ definida en su dominio natural $D$, \textbf{determine} y \textbf{grafique} $D$ y el conjunto de puntos $M$ de la curva de ecuaci�n $x=y^2-21$ donde $f$ queda definida.
	
	\item \blue{(08/06/19-T2-4) Dada $f:D\subset\R^2\rightarrow\R / f(x, y)=\ln{(y+x^2)}$ definida en su dominio natural $D$, \textbf{determine} y \textbf{grafique} $D$ y el conjunto de puntos $M$ de la curva de ecuaci�n $y=x^2-21$ donde $f$ queda definida.}
	
	\item (13/05/19-T?-1) \red{Este ejercicio incluye temas del TPIII} Dada $ f(x, y)=\sqrt{4-x^2-y^2}$ definida en su dominio natural $D$, \textbf{determine} y \textbf{grafique} $D$ y el conjunto de puntos $H$ para los cuales $f'_y(x,y)=1$.
	
	\item (11/05/19-T1-1) \red{Este ejercicio incluye temas del TPIII} Dada $ f(x, y)=\ln{(x^2y)}$ definida en su dominio natural $D$, \textbf{determine} y \textbf{grafique} $D$ y el conjunto de puntos $H$ para los cuales $f''_{xx}(x,y)-f''_{yy}(x,y)=0$.
	
	\item \blue{(11/05/19-T2-1) \red{Este ejercicio incluye temas del TPIII} Dada $ f(x, y)=\ln{(xy^2)}$ definida en su dominio natural $D$, \textbf{determine} y \textbf{grafique} $D$ y el conjunto de puntos $H$ para los cuales $f''_{xx}(x,y)-f''_{yy}(x,y)=0$.}
	
	\item (11/12/18-T1-4) Dada $f:D\subset\R^2\rightarrow\R / \vec{f}(x, y)=(\ln{(y-x^2)}, \sqrt{9-y-x^2})$ donde $D$ es el dominio natural de $\vec{f}$, \textbf{determine} y \textbf{grafique} el mencionado dominio $D$ y el conjunto  $H$ en cuyos puntos alguna de las componentes del campo resulte nula.
	
	\item \blue{(11/12/18-T2-4) Dada $f:D\subset\R^2\rightarrow\R / \vec{f}(x, y)=(\ln{(x-y^2)}, \sqrt{9-x-y^2})$ donde $D$ es el dominio natural de $\vec{f}$, \textbf{determine} y \textbf{grafique} el mencionado dominio $D$ y el conjunto  $H$ en cuyos puntos alguna de las componentes del campo resulte nula.}
	
	\item (17/11/18-T1-1) \red{Este ejercicio incluye temas del TPIII y del TPV} Siendo $f(x,y)=x^3-6xy+3xy^2$ con $(x,y)) \in \R^2$, \b{grafique} el conjunto de puntos $H^+$ del plano $xy$ para el cual el determinsnte hessiano $H(x,y)$ de $f$ resulta positivo \green{y \b{analice} en cu�les de dichos puntos la funci�n $f$ produce extremo local, \b{clasific�ndolo} y \b{calculando} su valor.}  \red{No resolver la parte en verde hasta llegara a extremos en el TP V.}
	
	\item \blue{(17/11/18-T2-1) \red{Este ejercicio incluye temas del TPIII y del TPV} Siendo $f(x,y)=y^3-6xy+3yx^2$ con $(x,y)) \in \R^2$, \b{grafique} el conjunto de puntos $H^+$ del plano $xy$ para el cual el determinsnte hessiano $H(x,y)$ de $f$ resulta positivo \green{y \b{analice} en cu�les de dichos puntos la funci�n $f$ produce extremo local, \b{clasific�ndolo} y \b{calculando} su valor.}  \red{No resolver la parte en verde hasta llegara a extremos en el TP V.}}
	
	\item (22/10/18-T?-5) Dada $f:D\subset\R^2\rightarrow\R / f(x, y)=\frac{\sqrt{x^2+y^2-1}}{\ln{(y-x)}}$, donde $D$ es el dominio natural de la funci�n. \textbf{Determine} y \textbf{grafique} el dominio $D$ y el conjunto   de nivel $0$ de $f$. \b{Indique} un ejemplo de punto exterior y otro de punto frontera de $D$.
	
	\item (20/10/18-T1-1) Sea $f:D\subset\R^2\rightarrow\R / f(x, y)=\frac{\ln{(x-y)}}{\sqrt{x-1}}$, donde $D$ es el dominio natural de la funci�n. \textbf{Determine} y \textbf{grafique} el dominio $D$ y el conjunto   de nivel $0$ de $f$. \b{Indique} un ejemplo de punto interior y otro de punto frontera de $D$.
	
	\item \blue{(20/10/18-T2-1) Sea $f:D\subset\R^2\rightarrow\R / f(x, y)=\frac{\ln{(y-x)}}{\sqrt{y-1}}$, donde $D$ es el dominio natural de la funci�n. \textbf{Determine} y \textbf{grafique} el dominio $D$ y el conjunto   de nivel $0$ de $f$. \b{Indique} un ejemplo de punto interior y otro de punto frontera de $D$.}
\end{enumerate}

\subsection{L�mites y continuidad}
\begin{enumerate}
	\item \red{Este ejercicio incluye temas del TPIII} Sea $f:\R^2 \rightarrow \R$ definida por:
	\[f(x,y)= \begin{cases}
	\frac{xy}{x^2+2y^2} & \text{ si } (x, y)\neq(0,0)\\
	0 & \text{ si } (x, y)=(0,0)
	\end{cases}\]
	\b{Analice} si $f$ es continua en $(0,0)$, en base a su conclusi�n \b{opine} con fundamento si la funci�n puede tener derivadas parciales continuas en un entorno de dicho punto.
	
\end{enumerate}
	
\end{document}